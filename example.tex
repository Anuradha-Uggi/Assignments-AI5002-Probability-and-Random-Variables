% UTF-8 encoding
\documentclass[9pt, dvipsnames]{beamer} %
% Beamer 设置
\usetheme[secheader]{Boadilla} % 使用的 Beamer 主题: Boadilla
\usecolortheme{beaver} % 使用的 Beamer 颜色:beaver
% 字体设置
\usefonttheme{professionalfonts} % professional 字体
% 其他 Package
\usepackage{times}
\usepackage{amsmath}
\usepackage{verbatim}
\usepackage{anyfontsize}
\usepackage{subcaption} % 子图片
\usepackage{graphicx} % 图片
\usepackage[export]{adjustbox}
\setbeamertemplate{caption}[numbered]
\newcounter{saveenumi}
\resetcounteronoverlays{saveenumi}
\usepackage[multidot]{grffile} % 允许文件名带多个点
\usepackage{tabularx} % 表格
\usepackage{tikz}
\usepackage{pgf}
\usepackage{textpos}
\usepackage{float}
\usepackage{wrapfig}
%%%%%%%%%%%%%%%%%%%
% 使用背景
\logo{\includegraphics[height=0.5cm]{iith.png}\hspace{12pt}\vspace {6pt}}

\pgfdeclareimage[height=\paperheight,width=\paperwidth]{bgimage}{vlc3.png}
\usebackgroundtemplate{\tikz\node[opacity=0.1,inner sep=0]{\pgfuseimage{bgimage}};}
%%%%%%%%%%%%%%%%%%%
%\usepackage{ctex} % xelatex 中文
\title{Study on VLC Channel Model Based on Poisson Stochastic Network Theory} % 标题

\author[Anuradha Uggi]{Hao Wu \inst{1} \and Qunjhen Fan\inst{2}}
%{\includegraphics[height=2cm,width=2cm]{iith.png}\\Anuradha Uggi\\ ee21resch01008}% 作者
\institute[IITH]{Presentation\\By\\Name:Anuradha Uggi\\I'd:EE21RESCH01008\\Course:AI5002\\Supervisor:Prof.GVV\\Indian Institute of Technology Hyderabad}

\date{\today} % 如果 Date 参数为空,自动显示当前日期
\begin{document}

    \everymath{\displaystyle}

    % 标题页
    \begin{frame}
        \titlepage % 根据上面信息生成标题
    \end{frame}

    \begin{frame}
        \frametitle{\textbf{Index}}
        \tableofcontents % 生成目录(如果为空,请编译两次)
    \end{frame}
    
%%%%%%%%%%%%%%%%%%%%%%%%%%%%%%%%%%%%%%%%%%%%%%%%%%%%%%%%%%%%%%%%%%
\section{VLC?}
%%%%%%%%%%%%%%%%%%%%%%%%%%%%%%%%%%%%%%%%%%%%%%%%%%%%%%%%%%%%%%%%%%
   
    \begin{frame}{\textbf {VLC?}}
      %  \frametitle{\textbf{VLC?}}
           \begin{columns}
          
             \column{0.4\textwidth}
             \begin{itemize}
              \item Message signals die out over longer distances 
              \item Requirement of Carriers
              \item Modem to Place message on a Carrier
            \end{itemize}
          \column{0.5\textwidth}
           \includegraphics[width=6.5cm,height=6cm,right]{vlc1.jpg}
           \captionof{figure}{\footnotesize Source:Internet}
          \end{columns}
   \end{frame}
   
   
%%%%%%%%%%%%%%%%%%%%%%%%%%%%%%%%%%%%%%%%%%%%%%%%%%%%%%%%%%%%%%%%%
 \section{Block Diagram of VLC Architecture}
%%%%%%%%%%%%%%%%%%%%%%%%%%%%%%%%%%%%%%%%%%%%%%%%%%%%%%%%%%%%%%%%%
   
    \begin{frame}{\textbf{Block Diagram of VLC Architecture}}
          
         \includegraphics[width=12.5cm,height=4cm,right]{blockD.jpeg}
         \captionof{figure}{\footnotesize Source:Conference Paper[2]}
        
    \end{frame}
    
%%%%%%%%%%%%%%%%%%%%%%%%%%%%%%%%%%%%%%%%%%%%%%%%%%%%%%%%%%%%%%%%
\subsection{Transmitter}
%%%%%%%%%%%%%%%%%%%%%%%%%%%%%%%%%%%%%%%%%%%%%%%%%%%%%%%%%%%%%%%%

    \begin{frame}{\textbf {Transmitter}}
      %  \frametitle{\textbf{VLC?}}
           \begin{columns}
          
             \column{0.4\textwidth}
             \begin{itemize}
              \item LD  and LED dual-functional
              \item Plank's Theory:$E_g=hf$
              \item Modulation
            \end{itemize}
          \column{0.5\textwidth}
           \includegraphics[width=7cm,height=6cm,right]{LED.png}
           \captionof{figure}{\footnotesize Source:Internet}
          \end{columns}
   \end{frame}

%%%%%%%%%%%%%%%%%%%%%%%%%%%%%%%%%%%%%%%%%%%%%%%%%%%%%%%%%%%%%%%%%%%%
\subsection{Receiver}
%%%%%%%%%%%%%%%%%%%%%%%%%%%%%%%%%%%%%%%%%%%%%%%%%%%%%%%%%%%%%%%%%%%%
    \begin{frame}{\textbf {Receiver}}
      %  \frametitle{\textbf{VLC?}}
           \begin{columns}
          
             \column{0.4\textwidth}
             \begin{itemize}
              \item Light2Electric Signal Conversion by Photo-detector 
              \item $R=\frac{e\times\eta}{h\times\nu}$ : Ratio of output current to Input Optical Power.
              
            \end{itemize}
          \column{0.5\textwidth}
           \includegraphics[width=6.5cm,height=6cm,right]{pinphotodiode.png}
           \captionof{figure}{\footnotesize Source:Internet}
          \end{columns}
   \end{frame}
   
%%%%%%%%%%%%%%%%%%%%%%%%%%%%%%%%%%%%%%%%%%%%%%%%%%%%%%%%%%%%%%%%%%%%
\subsection{Channel} 
%%%%%%%%%%%%%%%%%%%%%%%%%%%%%%%%%%%%%%%%%%%%%%%%%%%%%%%%%%%%%%%%%%%%
  
  \begin{frame}{\textbf {Channel}}
      %  \frametitle{\textbf{VLC?}}
           \begin{columns}
          
             \column{0.4\textwidth}
             \begin{itemize}
              \item Multi-Path Reflections 
              \item $\lambda_{VL}<\lambda_{RW} $
              \item VL interference
             \end{itemize}
          \column{0.5\textwidth}
           \includegraphics[width=7cm,height=6cm,right]{getImage.jpg}
           \captionof{figure}{\footnotesize Source:Internet}
          \end{columns}
   \end{frame}

%%%%%%%%%%%%%%%%%%%%%%%%%%%%%%%%%%%%%%%%%%%%%%%%%%%%%%%%%%%%%%%%%%%%%%
\subsection{Types of Channel}
%%%%%%%%%%%%%%%%%%%%%%%%%%%%%%%%%%%%%%%%%%%%%%%%%%%%%%%%%%%%%%%%%%%%%%

  
  \begin{frame}{\textbf {Types of Channel}}
      %  \frametitle{\textbf{VLC?}}
           \begin{columns}
          
             \column{0.4\textwidth}
             \begin{itemize}
              \item LoS and NLoS 
              \item Directed,Non-Directed and Hybrid Channels
             \end{itemize}
          \column{0.5\textwidth}
           \includegraphics[width=7cm,height=6cm,right]{channels.jpeg}
           \captionof{figure}{\footnotesize Source:Conference Paper[1]}
          \end{columns}
   \end{frame}
   
%%%%%%%%%%%%%%%%%%%%%%%%%%%%%%%%%%%%%%%%%%%%%%%%%%%%%%%%%%%%%%%%%%%%%%
\section{Statistical Model of VL Channel}
%%%%%%%%%%%%%%%%%%%%%%%%%%%%%%%%%%%%%%%%%%%%%%%%%%%%%%%%%%%%%%%%%%%%%%

   \begin{frame}{\textbf {Study of Channel}}
             \begin{itemize}
              \item Appropriate Rxd power level is desired in any communication System
              \item Interference may deteriorate Rxd power
              \item Increasing Txd power, Optimizing the LED Layout and decreasing FOV of Rx are remedies
              \item Study of Channel is needed to Compute Rxd power
              \item To study the Channel random distribution of RPs is needed
              \item Finding the best Layout of LED is the Motto of this paper
             
             \end{itemize}
   \end{frame}


%%%%%%%%%%%%%%%%%%%%%%%%%%%%%%%%%%%%%%%%%%%%%%%%%%%%%%%%%%%%%%%%%%%%%%
\subsection{Poisson Stochastic Network Theory Model(PSNTM)}
%%%%%%%%%%%%%%%%%%%%%%%%%%%%%%%%%%%%%%%%%%%%%%%%%%%%%%%%%%%%%%%%%%%%%%
  \begin{frame}{\textbf {Homogeneous Poisson Point Process}}
             \begin{itemize}
              \item Several RPs in space obey Poisson Distribution,so that the Network can be modelled as Poisson Stochastic Process 
              \item Due to several properties HPPP can well approximate the actual reflection environment\\ 
              \textbf{Poisson Distribution:}
              \begin{equation}
                  P(X=x)=\frac{e^{-\lambda}\lambda^{x}}{x!}
              \end{equation}
              \textbf{Poisson Point Process:}
              \begin{equation}
                  P(\phi(A) )=n)=\frac{e^{\lambda|A|}\lambda|A|^{n}}{n!}
              \end{equation}
             \end{itemize}
   \end{frame}
   
%%%%%%%%%%%%%%%%%%%%%%%%%%%%%%%%%%%%%%%%%%%%%%%%%%%%%%%%%%%%%%%%%%%%%%%
\subsection{Homogeneous Poisson Point Process}
%%%%%%%%%%%%%%%%%%%%%%%%%%%%%%%%%%%%%%%%%%%%%%%%%%%%%%%%%%%%%%%%%%%%%%%
 
   \begin{frame}{\textbf {Properties of HPPP}}
      %  \frametitle{\textbf{VLC?}}
           \begin{columns}
          
             \column{0.4\textwidth}
             \begin{itemize}
              \item If points in Space forms PP
              \item Subset of points is a rv with a Poisson Distribution
              \item Complete Independence
              \item Number of points in set A is a Poisson Rv with Parameter $\lambda |A|$
             \end{itemize}
          \column{0.5\textwidth}
           \includegraphics[width=6.5cm,height=6cm,right]{hppp.png}
           \captionof{figure}{\footnotesize Source:Python Simulation}
          \end{columns}
   \end{frame}

%%%%%%%%%%%%%%%%%%%%%%%%%%%%%%%%%%%%%%%%%%%%%%%%%%%%%%%%%%%%%%%%%%%%%%%
\subsection{SINR as Quality Metric}
%%%%%%%%%%%%%%%%%%%%%%%%%%%%%%%%%%%%%%%%%%%%%%%%%%%%%%%%%%%%%%%%%%%%%%%
 
   \begin{frame}{\textbf {SINR as Quality metric}}
      %  \frametitle{\textbf{VLC?}}
           \begin{columns}
          
             \column{0.4\textwidth}
             \begin{itemize}
              \item Optical signals are highly Incoherent
              \item Reflections are regarded as interference
              \item SINR to describe the quality of signal reception
              \item PPP is Translation Invariant so Rx at different positions is moved to origin
             \end{itemize}
          \column{0.5\textwidth}
           \includegraphics[width=6.5cm,height=6cm,right]{coherent.png}
           \captionof{figure}{\footnotesize Source:internet}
          \end{columns}
   \end{frame}
   
%%%%%%%%%%%%%%%%%%%%%%%%%%%%%%%%%%%%%%%%%%%%%%%%%%%%%%%%%%%%%%%%%%%%%%
%%%%%%%%%%%%%%%%%%%%%%%%%%%%%%%%%%%%%%%%%%%%%%%%%%%%%%%%%%%%%%%%%%%%%%
   \begin{frame}{\textbf {Signal to Interference+Noise Ratio}}
             \begin{itemize}
              \item SINR for a specific sender
              \begin{equation}
                  SINR_x(n)=\frac{P_x S_x(n)/l(|x|)}{N+I-P_xS_x(n)/l(|x|)}
              \end{equation}
              \begin{equation}
                  I=\sum_{x\in\phi}P_xS_x(n)/l(|x|)
              \end{equation}
               \item $|x|$ : distance from each sending end to receiving end
               \item $P_x$ : Transmitted power
               \item $S_x$ : Gain of each reflection point
               \item $N$ : Noise power
               \item $I$ : Total power received at Rx
               \item The path loss function 
               \begin{equation}
                   l(|x|)=(K|x|)^\beta
               \end{equation}
               \item $K>0$ and $4.5>\beta>2$
               
              
             \end{itemize}
   \end{frame}
%%%%%%%%%%%%%%%%%%%%%%%%%%%%%%%%%%%%%%%%%%%%%%%%%%%%%%%%%%%%%%%%%%%%%%%
%%%%%%%%%%%%%%%%%%%%%%%%%%%%%%%%%%%%%%%%%%%%%%%%%%%%%%%%%%%%%%%%%%%%%%%
 
   \begin{frame}{\textbf {Path Loss Function}}
          
             \begin{figure}
               \includegraphics[width=8.5cm,height=7cm]{pathloss.png}
               \captionof{figure}{\footnotesize Source:python simulation}
             \end{figure}
            
          
   \end{frame}
%%%%%%%%%%%%%%%%%%%%%%%%%%%%%%%%%%%%%%%%%%%%%%%%%%%%%%%%%%%%%%%%%%%%%%%
\section{Channel Model}
%%%%%%%%%%%%%%%%%%%%%%%%%%%%%%%%%%%%%%%%%%%%%%%%%%%%%%%%%%%%%%%%%%%%%%%
 
   \begin{frame}{\textbf {Transmission Model}}
          
             \begin{figure}
               \includegraphics[width=5cm,height=2cm]{transmission.jpeg}
               \captionof{figure}{\footnotesize Source:conference paper[1]}
             \end{figure}
             \begin{itemize}
             \item At the Tx end Photo-current signal $Y(t)$ can be expressed as:
             \begin{equation}
                 Y(t)=RX(t)\circledast h(t)+N(t)
             \end{equation}
              \item $R$ : Responsivity 
              \begin{equation}
                  R=\frac{I_{out}}{P_{in}}
              \end{equation}
              \begin{equation}
                  R=\frac{e\times \eta}{h\times \nu}
              \end{equation}
              \item $X(t)$ : light signal emitted by LED
              \item $h(t)$ : Channel shock effect
              \item $N(t)$ : AWGN noise
             \end{itemize}
          
   \end{frame}
   
%%%%%%%%%%%%%%%%%%%%%%%%%%%%%%%%%%%%%%%%%%%%%%%%%%%%%%%%%%%%%%%%%%%%%%
\subsection{Light Source Model}
%%%%%%%%%%%%%%%%%%%%%%%%%%%%%%%%%%%%%%%%%%%%%%%%%%%%%%%%%%%%%%%%%%%%%%
   \begin{frame}{\textbf {Light Source Model}}
             \begin{itemize}
              \item Lambertian model is used to simulate the emission of Light source
              \item Light intensity distribution of an LED can be expressed as:
              \begin{equation}
                  I(\theta)=I(0)cos^m(\theta),\theta\in(0,\frac{\pi}{2})
              \end{equation}
              \item $\theta$ is the beam exit angle
              \item $m$ is Lambert coefficient can be expressed as:
              \begin{equation}
                  m=-\frac{\ln 2}{\ln cos\theta_{1/2}}
              \end{equation}
              \item $\theta_{1/2}$ is the half power angle
              \item $I(0)$ is the central light intensity and can be expressed as:
              \begin{equation}
                  I(0)=\frac{m+1}{2\pi}P_s
              \end{equation}
              \item $P_s$ is the Txing power of light source
             \end{itemize}
   \end{frame}
%%%%%%%%%%%%%%%%%%%%%%%%%%%%%%%%%%%%%%%%%%%%%%%%%%%%%%%%%%%%%%%%%%%%%%%

%%%%%%%%%%%%%%%%%%%%%%%%%%%%%%%%%%%%%%%%%%%%%%%%%%%%%%%%%%%%%%%%%%%%%%%
 
   \begin{frame}{\textbf {Light Source Model simulation}}
          
             \begin{figure}
              % \includegraphics[width=5cm,height=2cm]
               %\captionof{figure}{\footnotesize Source:conference paper[1]}
             \end{figure}
   \end{frame}
   
%%%%%%%%%%%%%%%%%%%%%%%%%%%%%%%%%%%%%%%%%%%%%%%%%%%%%%%%%%%%%%%%%%%%%%%
\subsection{Impulse Response}
%%%%%%%%%%%%%%%%%%%%%%%%%%%%%%%%%%%%%%%%%%%%%%%%%%%%%%%%%%%%%%%%%%%%%%%
 
   \begin{frame}{\textbf {Impulse Response}}
          
             \begin{figure}
               \includegraphics[width=5cm,height=4cm]{impulse.jpg}
               \captionof{figure}{\footnotesize Source:conference paper[1]}
             \end{figure}
             \begin{itemize}
                \item Channel Impulse response=sum of Impulse responses of single Tx and Rx set.
                \item Impulse response consisting single LED and Rx is expressed as:
              \begin{equation}
                  h(t;S,R)=h^0(t;S,R)+\sum_{k=1}^{\infty}h^k(t;S,R)
              \end{equation}
             \end{itemize}
          
   \end{frame}
  
%%%%%%%%%%%%%%%%%%%%%%%%%%%%%%%%%%%%%%%%%%%%%%%%%%%%%%%%%%%%%%%%%%%%%%%%
%%%%%%%%%%%%%%%%%%%%%%%%%%%%%%%%%%%%%%%%%%%%%%%%%%%%%%%%%%%%%%%%%%%%%%%%
 
   \begin{frame}{\textbf {Impulse Response}}
             \begin{figure}
               \includegraphics[width=5cm,height=4cm]{photorx.jpg}
               \captionof{figure}{\footnotesize Source:conference paper[1]}
             \end{figure}      
   
             \begin{itemize}
                \item  Impulse response of the direct-view channel:
                \begin{equation}
                    h^0(t;S,R)=\frac{m+1}{2\pi}cos^m(\theta)d\Omega rect(\frac{\Psi}{FOV})\delta(t-\frac{d}{c})
                \end{equation}
                \item Solid angle of the Rx : 
                   $ d\Omega=cos\frac{A}{d^2}$
    
                \item $\Psi$ : receiving angle
                \item $d$ : distance between LED and Rx
                    
             \end{itemize}
          
   \end{frame}
   
%%%%%%%%%%%%%%%%%%%%%%%%%%%%%%%%%%%%%%%%%%%%%%%%%%%%%%%%%%%%%%%%%%%%%%%
\subsection{Received power}
%%%%%%%%%%%%%%%%%%%%%%%%%%%%%%%%%%%%%%%%%%%%%%%%%%%%%%%%%%%%%%%%%%%%%%%
 
   \begin{frame}{\textbf {Received power}}
          
             \begin{itemize}
                \item Continuous data connection in all the places in the room
                \item Improper placing of LEDs and too small FOV of Rx lead to blind spots.
                \item Received power of LED light source can be expressed as :
              \begin{equation}
                  P_r=H(0)P_t
              \end{equation}
              \item $P_r$ : Received power
              \item $P_t$ : Average transmitted power
             \end{itemize}
          
   \end{frame}
   
%%%%%%%%%%%%%%%%%%%%%%%%%%%%%%%%%%%%%%%%%%%%%%%%%%%%%%%%%%%%%%%%%%%%%%%
\subsection{Impact response $(H(0))$}
%%%%%%%%%%%%%%%%%%%%%%%%%%%%%%%%%%%%%%%%%%%%%%%%%%%%%%%%%%%%%%%%%%%%%%%
 
   \begin{frame}{\textbf {DC channel gain $(H(0))$}}
          
             \begin{itemize}
                \item DC channel gain from impact responce can be obtained as :
                \begin{equation}
                    H(0)=\begin{cases}
                        cos(\Psi)cos^m\Psi T_s(\Psi)g(\Psi) {\frac{A(m+1)}{2\pi d^2}}, & \text{$ 0\leq \Psi \leq \Psi_c $} \\
                        0, & \text{$\Psi>\Psi_c$}
                    
                          \end{cases}
                \end{equation}
                \item $T_s(\Psi)$ : optical filter gain
                \item $g(\Psi)$ : optical concentrator
              \begin{equation}
                  g(\Psi)=\begin{cases}
                           \frac{n^2}{sin^2\Psi_c},& \text{$0\leq\Psi\leq\Psi_c$}\\
                           0,& \text{$\Psi>\psi_c$}
                          \end{cases}
              \end{equation}
                \item $n$ is refractive index
             \end{itemize}
          
   \end{frame}
%%%%%%%%%%%%%%%%%%%%%%%%%%%%%%%%%%%%%%%%%%%%%%%%%%%%%%%%%%%%%%%%%%%%%%%
\section{Power fluctuation degree}
%%%%%%%%%%%%%%%%%%%%%%%%%%%%%%%%%%%%%%%%%%%%%%%%%%%%%%%%%%%%%%%%%%%%%%%
 
   \begin{frame}{\textbf {Variance of power distribution}}  
             
             \begin{equation}
                    D=\frac{1}{S}\iint_L[P_r(x,y)-\overline{P_r}]^2dxdy=\frac{1}{S}\sum_{i=1}^{N}\iint_L[P_tf(u_i,v_i;x,y)-\overline{P_r}]^2dxdy
              \end{equation}
              \begin{itemize}
                  \item Simulations for power fluctuation is performed for different layouts of LEDs
                  \item 4-LED are symmetrically distributed
                  \item Room dimensions:$5m\times 5m\times 3$
                  \item Rx is placed on a plane at a height h from the floor.
              \end{itemize}
              
                
    \end{frame}
    
%%%%%%%%%%%%%%%%%%%%%%%%%%%%%%%%%%%%%%%%%%%%%%%%%%%%%%%%%%%%%%%%%%%%%%%%
\section{Simulation Results}
%%%%%%%%%%%%%%%%%%%%%%%%%%%%%%%%%%%%%%%%%%%%%%%%%%%%%%%%%%%%%%%%%%%%%%%%
 
   \begin{frame}{\textbf {Impulse Response}}
           \begin{columns}
            \column{0.4\textwidth}
             \begin{figure}
               \includegraphics[width=6cm,height=7cm]{sim1.jpg}
               \captionof{figure}{\footnotesize Source:conference paper[1]}
             \end{figure}      
            \column{0.5\textwidth}
             \begin{figure}
               \includegraphics[width=6cm,height=7cm]{sim2.jpg}
               \captionof{figure}{\footnotesize Source:conference paper[1]}
             \end{figure}  
          \end{columns}  
          
   \end{frame}
   
%%%%%%%%%%%%%%%%%%%%%%%%%%%%%%%%%%%%%%%%%%%%%%%%%%%%%%%%%%%%%%%%%%%%%%%
\section{Applications}
%%%%%%%%%%%%%%%%%%%%%%%%%%%%%%%%%%%%%%%%%%%%%%%%%%%%%%%%%%%%%%%%%%%%%%%
 
   \begin{frame}{\textbf {Applications}}  
          
        \begin{itemize}
            \item By locating RPs Easedropping can be mitigated to add more security.
            \item Effective channel modelling results in reducing the effect of Interference which increases the SINR or quality of signal reception.
        \end{itemize}
              
                
    \end{frame}
   
%%%%%%%%%%%%%%%%%%%%%%%%%%%%%%%%%%%%%%%%%%%%%%%%%%%%%%%%%%%%%%%%%%%%%%%
\section{Conclusion}
%%%%%%%%%%%%%%%%%%%%%%%%%%%%%%%%%%%%%%%%%%%%%%%%%%%%%%%%%%%%%%%%%%%%%%%
 
   \begin{frame}{\textbf {Conclusion}}  
        The scheme proposed in this paper can be used to obtain the optimal layout of LED lamp in visible light communication system.
              
                
    \end{frame}
    \begin{frame}%%     2
          \begin{center}
             {\fontsize{40}{50}\selectfont Thank You!}
          \end{center}
     \end{frame}
    
\end{document}