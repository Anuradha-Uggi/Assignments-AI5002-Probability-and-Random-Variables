\documentclass[journal,12pt,twocolumn]{IEEEtran}
\usepackage{longtable}
\usepackage{amsthm}
\usepackage{graphics}
\usepackage{mathrsfs}
\usepackage{txfonts}
\usepackage{stfloats}
\usepackage{pgfplots}
\usepackage{cite}
\usepackage{cases}
\usepackage{mathtools}
\usepackage{caption}
\usepackage{enumerate}
\usepackage{enumitem}
\usepackage{amsmath}
\usepackage[utf8]{inputenc}
\usepackage[english]{babel}
\usepackage{multicol}
\usepackage{multirow}
\usepackage{mathtools}
\usepackage{gensymb}
\usepackage{amssymb}
\usepackage{pgfplots}
\usepackage{hyperref}
\usepackage{listings}
\usepackage{color}
\usepackage{array}
\usepackage{calc}
\usepackage{ifthen}
\usepackage{hhline}
\lstset{
%language=C,
frame=single,
breaklines=true,
columns=fullflexible
}

\title{Probability\&RV \\ Assignment-09}
\author{Anuradha U-ee21resch01008}
\date{\today}

\begin{document}
\maketitle
\newpage
\bigskip
\renewcommand{\thefigure}{\theenumi}
\renewcommand{\thetable}{\theenumi}


\textbf{Download Latex code from}
\begin{lstlisting}
https://github.com/Anuradha-Uggi/Assignments-AI5002-Probability-and-Random-Variables/blob/main/Prob_ass09/rvsp_9.tex
\end{lstlisting}
\section{\textbf{QUESTION(UGC NET 2019,Q-108)}}

Suppose $X_i$=X$_1$,X$_2$,....,X$_n$ are i.i.d \\ Uniform ($\theta$,2$\theta$),$\theta>$ 0. Let X$_{(1)}$=min\{X$_1$,$X_2$,...,X$_n$\} and 
X$_{(n)}$=max\{X$_1$,....,X$_n$\}.then which of the following statements are correct.\\

\begin{enumerate}
    \item (\ X$_{(1)}$,X$_{(n)}$)\ is jointly sufficient and complete for $\theta$
    \item (\ X$_{(1)}$,X$_{(n)}$)\ is jointly sufficient but not complete for
    $\theta$
    \item $\frac{X_{(n)}}{2}$ is maximum likelihood estimate for $\theta$
    \item X$_{(1)}$ is maximum likelihood estimate for $\theta$
    
\end{enumerate}
\section{\textbf{Basic Definitions}}
\textbf{Sufficient Statistic:}\\

Given X i.i.d Data conditioned on an unknown parameter $\theta$, T(X) is called sufficient statistic for $\theta$ if its values contains all the information needed to compute any estimate of the parameter (Maximum likelihood estimate). according to Fisher$-$Neymen Factorization PDF is \\

\begin{equation}
    f(X;\theta)=h(X)S(\theta,T(X))
\end{equation}

where h(X) is a constant and S($\theta$,T(X)) is a function through which $\theta$ will interact to X only through T(X).\\ \\
\textbf{Statistic Completeness:}\\ \\
T(X) is said to be complete for $\theta$ if for every measurable function g;
if 
\begin{equation}
    {E_\theta}(g(T))=0 
\end{equation}
for all $\theta$ then
\begin{equation}
    P_\theta(g(T)=0)=1 
\end{equation}
for all $\theta$.\\
\section{\textbf{SOLUTION}}
Given

\begin{equation}
      X_{(1)} =min\{X_1,X_2,....,X_n\}=min\{X_i\}
\end{equation}

\begin{equation}
    X_{(n)}=max\{X_1,X_2,....,X_n\}=max\{X_i\}
\end{equation}

\begin{equation}
    X_i=\{X_1,X_2,....,X_n\}\sim U(\theta,2\theta)
\end{equation}

\begin{equation}
    P(X_i)=\dfrac{1}{\theta}
\end{equation}
Statistic

\begin{equation}
    T(X)=(min\{X\},max\{X\})
\end{equation}

X are i.i.d so Likelihood is given by 
\begin{equation}
    f_X(x)=\frac{1}{\theta^n}1_{\{\theta\leq X_i \leq2\theta\}}
\end{equation}

equation (7) can be split into

\begin{equation}
    f_X(x)=\frac{1}{\theta^n}1_{\{\theta\leq min\{X_i\}}1_{max\{ X_i\}\leq2\theta}
\end{equation}
from equation (8) 

\begin{equation}
    h(X)=1
\end{equation}
 which is constant and
 
\begin{equation}
    S_{(\theta,2\theta)}(X)=\frac{1}{\theta^n}
\end{equation}

 which is function of only $\theta$.\\
therefore $T(min\{X_i\},max\{X_i\})$ is jointly sufficient to define $\theta$ thus Sufficient statistic.\\ \\
Let 

\begin{equation}
     g(T)=max\{X_i\}-max\{X_i\}
\end{equation}

\begin{equation}
    E[g(T)]=E[max\{X_i\}-min\{X_i\}]=\int (max\{X_i\}-min\{X_i\})\frac{1}{\theta^n}dx
\end{equation}

from equation (12) its clear that
\begin{equation}
    max\{X_i\}-min\{X_i\}\neq0
\end{equation}

for all $\theta$ therefore 
\begin{equation}
    P(\ max\{X_i\}-min\{X_i\}=0 )\ \neq1
\end{equation}

therefore $T(X_{(n)},X_{(1)})$ is Jointly sufficient but not complete for $\theta$. \\ \\
\textbf{Maximum Likelihood Estimate(MLE) :}\\ \\
Likelihood can be written as 

\begin{equation}
    f_\theta(X_1,X_2,..,X_n)=\frac{1}{\theta^n}I\left( \frac{max     \{X_i\}}{2}\leq\theta\leq min\{X_i\}\right)
\end{equation}

the MLE is the statistic that maximizes the liklihood.from equation (7) liklihood is a decreasing function of $\theta$.therefore MLE of $\theta$ is 

\begin{equation}
    \theta=\frac{X_{(n)}}{2}=\frac{max\{X_i\}}{2}
\end{equation}

\section{\textbf{CONCLUTION}}
From above observations option (2) and option (3) holds.













\end{document}