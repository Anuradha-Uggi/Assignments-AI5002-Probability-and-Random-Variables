\documentclass[journal,12pt,twocolumn]{IEEEtran}
\usepackage{longtable}
\usepackage{amsthm}
\usepackage{graphics}
\usepackage{mathrsfs}
\usepackage{txfonts}
\usepackage{stfloats}
\usepackage{pgfplots}
\usepackage{cite}
\usepackage{cases}
\usepackage{mathtools}
\usepackage{caption}
\usepackage{enumerate}
\usepackage{enumitem}
\usepackage{amsmath}
\usepackage[utf8]{inputenc}
\usepackage[english]{babel}
\usepackage{multicol}
\usepackage{multirow}
\usepackage{mathtools}
\usepackage{gensymb}
\usepackage{amssymb}
\usepackage{pgfplots}
\usepackage{hyperref}
\usepackage{listings}
\usepackage{color}
\usepackage{array}
\usepackage{calc}
\usepackage{ifthen}
\usepackage{hhline}
\lstset{
%language=C,
frame=single,
breaklines=true,
columns=fullflexible
}

\title{Probability\&RV \\ Assignment-08}
\author{Anuradha U-ee21resch01008}
\date{\today}

\begin{document}
\maketitle
\newpage
\bigskip
\renewcommand{\thefigure}{\theenumi}
\renewcommand{\thetable}{\theenumi}


\textbf{Download Latex code from}
\begin{lstlisting}
https://github.com/Anuradha-Uggi/Assignments-AI5002-Probability-and-Random-Variables/blob/main/Prob_ass08/rvsp_8.tex
\end{lstlisting}
\textbf{Download Python code from}
\begin{lstlisting}
https://github.com/Anuradha-Uggi/Assignments-AI5002-Probability-and-Random-Variables/blob/main/Prob_ass08/rvsp_8.py
\end{lstlisting}
\section{\textbf{QUESTION(GATE-Q17)}}
The input X to the binary Symmetric Channel(BSC) shown in fig.1 is '1' with probability 0.8. The cross-over probability is $\frac{1}{7}$. if the received bit Y=0,the conditional probability that '1' was transmitted is......

\section{\textbf{SOLUTION}}
Given 
\begin{equation}
    P(Y=0/X=0)=P(Y=1/X=1)= \frac{6}{7}
\end{equation}
\begin{equation}
    P(Y=0/X=1)=P(Y=1/X=0)=\frac{1}{7}
\end{equation}
we know that
\begin{equation}
    P(X\cap Y)=P(Y\cap X) 
\end{equation}
Above equation can also be written as
\begin{equation}
    P(X/Y)P(Y)=P(Y/X)P(X)
\end{equation}

Therefore
\begin{equation}
    P(X=1/Y=0)=\frac{P(Y=0/X=1)P(X=1)}{P(Y=0)}
\end{equation}
From the given data
\begin{equation}
    P(Y=0)=P(Y=0/X=0)P(X=0)+P(Y=0/X=1)P(X=1)
\end{equation}
\begin{equation}
    P(Y=0)= \frac{6}{7}\times 0.2+\frac{1}{7}\times 0.8=\frac{2}{7}
\end{equation}
we have
\begin{enumerate}
    \item $P(Y=0/X=1)=\frac{1}{7}$
    \item $P(X=1)=0.8$
    \item $P(Y=0)=\frac{2}{7}$
\end{enumerate}
Substituting above values in equation (5) results

\begin{equation}
    P(X=1/Y=0)=\frac{0.8}{2}=0.4
\end{equation}

\begin{figure}
    \centering
    \includegraphics[width=0.5\textwidth]{rvsp8.png}
\end{figure}

\section{\textbf{CONCLUSION}}
probability that X=1 is transmitted given that Y=0 is received is
\begin{equation}
    P(X=1/Y=0)=0.4
\end{equation}





\end{document}
