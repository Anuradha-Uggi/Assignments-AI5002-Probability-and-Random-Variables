\documentclass[journal,12pt,twocolumn]{IEEEtran}
\usepackage{longtable}
\usepackage{amsthm}
\usepackage{graphics}
\usepackage{mathrsfs}
\usepackage{txfonts}
\usepackage{stfloats}
\usepackage{pgfplots}
\usepackage{cite}
\usepackage{cases}
\usepackage{mathtools}
\usepackage{caption}
\usepackage{enumerate}
\usepackage{enumitem}
\usepackage{amsmath}
\usepackage[utf8]{inputenc}
\usepackage[english]{babel}
\usepackage{multicol}
\usepackage{multirow}
\usepackage{mathtools}
\usepackage{gensymb}
\usepackage{amssymb}
\usepackage{pgfplots}
\usepackage{hyperref}
\usepackage{listings}
\usepackage{color}
\usepackage{array}
\usepackage{calc}
\usepackage{ifthen}
\usepackage{hhline}
\lstset{
%language=C,
frame=single,
breaklines=true,
columns=fullflexible
}

\title{Probability\&RV \\ Assignment-05}
\author{U Anuradha-ee21resch01008}
\date{\today}

\begin{document}
\maketitle
\newpage
\bigskip
\renewcommand{\thefigure}{\theenumi}
\renewcommand{\thetable}{\theenumi}
\textbf{download Python code from}
\begin{lstlisting}
 https://github.com/Anuradha-Uggi/Assignments-AI5002-Probability-and-Random-Variables/blob/main/Prob_ass05/rvsp_ass_5.py
\end{lstlisting}
\textbf{Download Latex code from}
\begin{lstlisting}
https://github.com/Anuradha-Uggi/Assignments-AI5002-Probability-and-Random-Variables/blob/main/Prob_ass05/rvsp_5.tex
\end{lstlisting}
\section{\textbf{QUESTION}}
A bag consists of 10 balls each marked with one of the digits 0 to 9. If four balls are drawn successively with replacement from the bag,
what is the probability that none is marked with the digit 0?
\section{\textbf{SOLUTION}}
\textbf{Problem  Description:}\\ \\
(i) An Urn consists 10 Balls each marked with any of the digits from 0 to 9.\\
(ii) After every Draw the Ball is replaced in the Urn.\\
(iii) 4 Balls needs to be Drawn from the Urn by replacing Balls after every Draw.\\
(iv) All Balls Drawn should be marked with non-zero digits.\\ \\
\textbf{Computation:}\\ \\
Now we should Compute Probability of Drawing 4 Balls from the Urn marked with non-zero digits with Replacement.\\
Here we Define Two Random Variables X and Y. X Describes whether the Ball Drawn is marked with zero or non-zero digit.and Y Describes the number of balls marked non-zero digit Drawn from the Urn.\\
where
\begin{equation*}
    P(X=0) 
\end{equation*}
is the Probability of Drawing a ball.and
\begin{equation*}
    P(X=1)
\end{equation*}
is the Probability that Ball Drawn is marked with non-zero digit.\\
we know that
\begin{equation*}
    P(X=0)=\frac{1}{10}
\end{equation*}
where Sample space consists all 10 possibilities and among  them the Favourable Outcome is Drawing any ball.similarly
\begin{equation*}
    P(X=1)=1-P(X=0)=\frac{9}{10}
\end{equation*}
where we have Digits from 0 to 9.one ball among 10 is generally marked with 0.so now Sample Space remains same but Number of Favourable Outcomes will change to any digit except zero.therefore it becomes 9.so 9 over 10.\\  \\
since balls drawn are replaced,probability of success or failure remains same in all Trials.and the Trial is Associated with two outcomes either success or failure.so the Random Variable X is a Bernoulli Random Variable.where 
\begin{equation*}
    P(X=1) = \frac{9}{10}
\end{equation*}
is the success Probability.and
\begin{equation*}
    P(X=0)=1-P(X=1)=\frac{1}{10}
\end{equation*}
is the failure probability.\\
Now final Probability can be obtained as below.\\ 
random variable Y can take values \\
Y=0\\Y=1\\Y=2\\Y=3\\Y=4.\\
let
\begin{equation*}
    q=P(X=1)
\end{equation*}
\begin{equation*}
    1-q=P(X=0)
\end{equation*}
then
\begin{equation*}
    P(Y=0)=\binom{4}{0}(1-q)^4
\end{equation*}
\begin{equation*}
    P(Y=1)=\binom{4}{1}(q)(1-q)^3 
\end{equation*}
\begin{equation*}
    P(Y=2)=\binom{4}{2}(q)^2(1-q)^2
\end{equation*}
\begin{equation*}
    P(Y=3)=\binom{4}{3}(q)^3(1-q)^1
\end{equation*}
\begin{equation*}
    P(Y=4)=\binom{4}{4}(q)^4
\end{equation*}
Now we can state our Final Computed Probability that 4 balls are drawn and none of them is marked with zero-digit is 
\begin{equation*}
    P(Y=4)=\left(\frac{9}{10}\right)^4
\end{equation*}
\textbf{Generalization}\\
Above task can be Generalized using Binomial Distribution as
\begin{equation*}
    P_Y=\binom{n}{Y}(q)^Y(1-q)^{n-Y}
\end{equation*}
where n=4.\\
if n=1000 then $q=\frac{999}{1000}$.
\section{\textbf{CONCLUSION}}
the probability that among 4 balls Drawn  none is marked with the digit 0 is
\begin{equation*}
    P(Y=4)=\left(\frac{9}{10}\right)^4
\end{equation*}
\end{document}