\documentclass[journal,12pt,twocolumn]{IEEEtran}
\usepackage{longtable}
\usepackage{amsthm}
\usepackage{graphics}
\usepackage{mathrsfs}
\usepackage{txfonts}
\usepackage{stfloats}
\usepackage{pgfplots}
\usepackage{cite}
\usepackage{cases}
\usepackage{mathtools}
\usepackage{caption}
\usepackage{enumerate}
\usepackage{enumitem}
\usepackage{amsmath}
\usepackage[utf8]{inputenc}
\usepackage[english]{babel}
\usepackage{multicol}
\usepackage{multirow}
\usepackage{mathtools}
\usepackage{gensymb}
\usepackage{amssymb}
\usepackage{pgfplots}
\usepackage{hyperref}
\usepackage{listings}
\usepackage{color}
\usepackage{array}
\usepackage{calc}
\usepackage{ifthen}
\usepackage{hhline}
\lstset{
%language=C,
frame=single,
breaklines=true,
columns=fullflexible
}

\title{Probability\&RV \\ Assignment-05}
\author{U Anuradha-ee21resch01008}
\date{\today}

\begin{document}
\maketitle
\newpage
\bigskip
\renewcommand{\thefigure}{\theenumi}
\renewcommand{\thetable}{\theenumi}
\textbf{download Python code from}
\begin{lstlisting}
 https://github.com/Anuradha-Uggi/Assignments-AI5002-Probability-and-Random-Variables/blob/main/Prob_ass05/rvsp_ass_5.py
\end{lstlisting}
\textbf{Download Latex code from}
\begin{lstlisting}
https://github.com/Anuradha-Uggi/Assignments-AI5002-Probability-and-Random-Variables/blob/main/Prob_ass05/rvsp_5.tex
\end{lstlisting}
\section{\textbf{QUESTION}}
A bag consists of 10 balls each marked with one of the digits 0 to 9. If four balls are drawn successively with replacement from the bag,
what is the probability that none is marked with the digit 0?
\section{\textbf{SOLUTION}}
\textbf{Problem  Description:}
\begin{enumerate}
    \item  10 Balls in an Urn  marked with  digits from 0 to 9\\
    \item After every Draw the Ball should be replaced  \\
    \item 4 Balls needs to be Drawn\\
    \item All Balls Drawn should be marked with non-zero digit.\\
\end{enumerate}
  
\textbf{Computation:}\\ 
Here we Define Two Random Variables X and Y. \\
\begin{itemize}
    \item X Describes whether the Ball Drawn is marked with zero or non-zero digit\\
    \item Y Describes the number of balls Drawn that are marked with non-zero digit \\
    \item P(X=0) is the Probability of Drawing any ball \\
    \item  P(X=1) is the Probability that Ball Drawn is marked with non-zero digit.\\
\end{itemize}
we know that
\begin{equation}
    P(X=0)=\frac{1}{10}
\end{equation}
where Sample space consists all 10 possibilities and among  them the Favourable Outcome is Drawing any ball.\\
similarly
\begin{equation}
    P(X=1)=1-P(X=0)=\frac{9}{10}
\end{equation}
One ball among 10 is generally marked with 0.so now Sample Space remains same but Number of Favourable Outcomes becomes 9.\\  \\
Since Trials are  Bernoulli Trials Hence the Random Variable X.
random variable Y can take values from 0 to 4.
let
\begin{equation}
    q=P(X=1)
\end{equation}
\begin{equation}
    1-q=P(X=0)
\end{equation}
then
\begin{itemize}
    \item $P(Y=0)=\binom{4}{0}(1-q)^4$ \\
    \item $P(Y=1)=\binom{4}{1}(q)(1-q)^3$ \\
    \item $P(Y=2)=\binom{4}{2}(q)^2(1-q)^2 $ \\
    \item $P(Y=3)=\binom{4}{3}(q)^3(1-q)^1$ \\
    \item $P(Y=4)=\binom{4}{4}(q)^4 $ \\
\end{itemize}
 Probability that 4 balls are drawn and none of them is marked with zero-digit is 
\begin{equation}
    P(Y=4)=\left(\frac{9}{10}\right)^4
\end{equation}
\textbf{Generalization}\\
Above task can be Generalized using Binomial Distribution as
\begin{equation}
    P_Y=\binom{n}{Y}(q)^Y(1-q)^{n-Y}
\end{equation}
where n=4.\\
if n=1000 then $q=\frac{999}{1000}$.
\section{\textbf{CONCLUSION}}
the probability that among 4 balls Drawn  none is marked with the digit 0 is
\begin{equation}
    P(Y=4)=\left(\frac{9}{10}\right)^4
\end{equation}
\end{document}