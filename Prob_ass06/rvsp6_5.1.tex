\documentclass[journal,12pt,twocolumn]{IEEEtran}
\usepackage{longtable}
\usepackage{amsthm}
\usepackage{graphics}
\usepackage{mathrsfs}
\usepackage{txfonts}
\usepackage{stfloats}
\usepackage{pgfplots}
\usepackage{cite}
\usepackage{cases}
\usepackage{mathtools}
\usepackage{caption}
\usepackage{enumerate}
\usepackage{enumitem}
\usepackage{amsmath}
\usepackage[utf8]{inputenc}
\usepackage[english]{babel}
\usepackage{multicol}
\usepackage{multirow}
\usepackage{mathtools}
\usepackage{gensymb}
\usepackage{amssymb}
\usepackage{pgfplots}
\usepackage{hyperref}
\usepackage{listings}
\usepackage{color}
\usepackage{array}
\usepackage{calc}
\usepackage{ifthen}
\usepackage{hhline}
\lstset{
%language=C,
frame=single,
breaklines=true,
columns=fullflexible
}

\title{Probability\&RV \\ Assignment-06}
\author{Anuradha U-ee21resch01008}
\date{\today}

\begin{document}
\maketitle
\newpage
\bigskip
\renewcommand{\thefigure}{\theenumi}
\renewcommand{\thetable}{\theenumi}
\textbf{download Python code from}
\begin{lstlisting}
https://github.com/Anuradha-Uggi/Assignments-AI5002-Probability-and-Random-Variables/blob/main/Prob_ass06/rvsp6_51.py
\end{lstlisting}
\textbf{download Latex code from}
\begin{lstlisting}
https://github.com/Anuradha-Uggi/Assignments-AI5002-Probability-and-Random-Variables/blob/main/Prob_ass06/rvsp6_5.1.tex
\end{lstlisting}
\section{\textbf{QUESTION(PROB,5.1)}}
It is given that in a group of 3 students, the probability of 2 students not having the same birthday is 0.992.What is the probability that 2 students have the same birthday?

\section{\textbf{SOLUTION}}
Let X is a random variable indicates number of people sharing their birthday and Y is a random variable indicates X people sharing or not sharing their birthdays.\\ \\
\textbf{Given Data:} \\
Probability of 2 people not sharing the Birthday is
\begin{equation}
    P(Y=0/X=2)=0.992
\end{equation}
from the Axioms of Probability we can say that
\begin{equation}
    P(Y=0/X)+P(Y=1/X)=1
\end{equation}
from above equation (2)\\
Probability that 2 students among 3 have same birthday is 
\begin{equation}
    P(Y=1/X=2)=1-P(Y=0/X=2)
\end{equation}
\begin{equation}
    P(Y=1/X=2)=1-0.992=0.008
\end{equation}
\\
\textbf{Generalization:}\\
\begin{itemize}
    \item Let us consider non leap years having 365 days/year. 
    \item Then Probability that k among n people share birthday is given by
\end{itemize}

\begin{equation}
    P(Y=1/X=k)= \frac{\binom{n}{k}\times \binom{365}{n+1-k}\times (n+1-k)!}{365^n}
\end{equation}
\\
\textbf{Conclusion:}\\
Probability that 2 among 3 people having same birthday is 0.008.







\end{document}