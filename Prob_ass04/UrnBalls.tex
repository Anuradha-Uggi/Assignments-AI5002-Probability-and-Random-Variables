\documentclass[journal,12pt,twocolumn]{IEEEtran}
\usepackage{longtable}
\usepackage{amsthm}
\usepackage{graphics}
\usepackage{mathrsfs}
\usepackage{txfonts}
\usepackage{stfloats}
\usepackage{pgfplots}
\usepackage{cite}
\usepackage{cases}
\usepackage{mathtools}
\usepackage{caption}
\usepackage{enumerate}
\usepackage{enumitem}
\usepackage{amsmath}
\usepackage[utf8]{inputenc}
\usepackage[english]{babel}
\usepackage{multicol}
\usepackage{multirow}
\usepackage{mathtools}
\usepackage{gensymb}
\usepackage{amssymb}
\usepackage{pgfplots}
\usepackage{hyperref}
\usepackage{listings}
\usepackage{color}
\usepackage{array}
\usepackage{calc}
\usepackage{ifthen}
\usepackage{hhline}
\lstset{
%language=C,
frame=single,
breaklines=true,
columns=fullflexible
}

\title{Probability\&RV \\ Assignment-04}
\author{U Anuradha-ee21resch01008}
\date{\today}

\begin{document}
\maketitle
\newpage
\bigskip
\renewcommand{\thefigure}{\theenumi}
\renewcommand{\thetable}{\theenumi}
\textbf{download Python code from}
\begin{lstlisting}
 https://github.com/Anuradha-Uggi/Assignments-AI5002-Probability-and-Random-Variables/blob/main/Prob_ass04/rvsp_urn_balls.py
\end{lstlisting}
\textbf{Download Latex code from}
\begin{lstlisting}
https://github.com/Anuradha-Uggi/Assignments-AI5002-Probability-and-Random-Variables/blob/main/Prob_ass04/UrnBalls.tex
\end{lstlisting}
\section{\textbf{QUESTION}}
Six balls are drawn successively from an
urn containing 7 red and 9 black balls. Tell
whether or not the trials of drawing balls are
Bernoulli trials when after each draw the ball
drawn is\\
(i) replaced\\
(ii)not replaced in the urn.
\section{\textbf{SOLUTION}}
Properties to be satisfied if a trial needs to be a bernoulli trial:\\
\begin{enumerate}
    \item Number of trials should be finite.
    \item each trial should have utcomes of success and failure.\
    \item if P is the success probability then failure probability should be 1-P
    \item probability of success should not vary with trial
\end{enumerate}

\textbf{Case(i):Replaced} \\ 
Number of red balls = 7\\
Number of black balls = 9\\
let X be the random variable and 
\begin{itemize}
    \item X=1 is success which is Drawing red ball
    \item X=0 is Failure which is Drawing black ball
\end{itemize}
Success Probability 
\begin{equation}
    P(X=1) = \frac{7}{16} 
\end{equation}
Failure Probability
\begin{equation}
     P(X=0) = \frac{9}{16} = 1-P(X=1)
\end{equation}
 Success Probability is  constant for all Trials.
 as X satisfies all properties of Bernoulli therefore Trials are Bernoulli Trials.\\ \\
\textbf{Case(ii):Not Replaced} \\ \\
In this case Success Probability is 
\begin{equation}
    P(X=1) =\frac{7}{16}
\end{equation}
 for Second Trial 
\begin{equation}
     P(X=1) = \frac{6}{15} 
\end{equation} 
 Corresponding Failure Probabilities are 
 \begin{equation}
     P(X=0) = \frac{9}{16}
 \end{equation} and for 2nd trial 
 \begin{equation}
     P(X=0) = \frac{8}{15}
 \end{equation}  
  probability of success and corresponding failure is varying with trials therefore these are not Bernoulli Trials. 
\section{\textbf{CONCLUSION}}
\begin{itemize}
    \item Case(i):Trials are Bernoulli Trials 
    \item Case(ii): Trials are not Bernoulli Trials
\end{itemize}








\end{document}